\section{Développement d'un proxy}

\subsection{Version standard du proxy}

\paragraph{}
La première version du proxy n'est que l'adaptation de l'algorithme au langage. Il n'accepte qu'un seul client à la fois. Toute autre demande durant le traitement d'un client est ignorée. 
\paragraph{}
Certains morceaux de code redondant ont été transformés en fonction dans les fichiers util.c et socketutil.c du dossier utils. Le premier rassemble les fonctions de traitement de chaînes de caractère et des fichiers de log, tandis que le second regroupe les fonctions de traitement de socket (création de socket, ajout d'un client, ...).
\paragraph{}
Afin d'avoir un aperçu global du bon fonctionnement du proxy, un affichage exhaustif de son comportement a été réalisé (connexion/déconnexion client, requête reçue, fermeture de la connexion avec le serveur web, réponse envoyée, ...). 
\paragraph{}
On affiche aussi l'adresse Ip et le port de lancement. On remarque qu'étant donné que le serveur attend autant de l'IPv4 que de l'IPv6, son adresse est "::". 

\subsection{Version multi-utilisateur}
\paragraph{}
Afin de faciliter les modifications de code dans les deux versions par la suite, de nouvelles fonctions ont été ajoutées à util.c et socketutil.c pour y mettre les morceaux de code communs (initialisation des sockets d'écoute, ...).
\paragraph{}
Dans cette seconde version, afin de gérer plusieurs clients, nous utilisons un tableau de descripteurs pour les sockets clients et pour les sockets web. Nous initialisons par défaut ces tableaux au nombre maximal de descripteur d'un set de descripteur (soit 1024) étant donné que le select ne pourra pas en gérer plus (on considère que le nombre de descripteurs déjà utilisé pour les sockets d'écoute et les entrée/sortie standard est négligeable). \linebreak
Afin de faciliter la correspondance entre les sockets clients et les websockets, on les associera par le même indice (un client d'indice i attend la réponse de la websocket d'indice i).

\subsection{Fichiers de logs}
\paragraph{}
Afin d'avoir un meilleur aperçu de l'utilisation du proxy, les deux versions renseignent des fichiers de logs (qui se trouvent dans le dossier logs). 
\paragraph{}
Le premier fichier (log\_visits) enregistre l'adresse IP d'un client qui se connecte au proxy. Si celui-ci n'est pas déjà dans le fichier de log, alors on l'ajoute en mettant le compteur de visite à 1. Si le client se trouve déjà dans le fichier, le compteur est incrémenté. Ceci permet d'avoir une vue globale de ses visiteurs et du nombre de leurs visites, et pourrait aussi permettre d'identifier les clients abusifs. 
\paragraph{}
Le second fichier (log\_requests) enregistre chaque requête effectuée auprès du proxy avec l'IP et le port d'origine de la demande. Afin de trouver l'IP d'un client à partir de son descripteur, la fonction \textit{getpeername} a été utilisée. Ce fichier permet d'avoir une liste exhaustive des requêtes faites par le proxy et ainsi d'isoler les requêtes illégales et/ou les clients effectuant trop de requêtes.

\subsection{Ce qui pourrait être encore fait : le time-out}
\paragraph{}
On peut remarquer lors de l'utilisation du proxy que les déconnexions et reconnexions sont très fréquentes. Ceci est sans doute due au fait que les connexions ne soient pas conservées. On pourrait imaginer vérifier dans la réponse la durée de vie de la requête (keep-alive) afin de faire perdurer les connexions qui le doivent. 

\subsection{Version HTTPS}
\paragraph{}
Nous tenons à remercier à Nicolas \textsc{Bédrine} pour cette partie. En effet, les forums sur lesquels nous sommes tombés proposaient d'utiliser la bibliothèque C openSSL. Cependant nous n'avons pas réussi à implémenter le code de manière fonctionnelle avec cette méthode. Avec le recul, cette bibliothèque est sans doute plus utile pour un serveur ou un client HTTPS que pour un proxy. C'est donc Nicolas qui nous a expliqué le fonctionnement de HTTPS. \linebreak
Nous suivons donc le schéma suivant : lorsque nous recevons un paquet CONNECT, nous savons qu'un client veut initier une connexion HTTPS. Pour ce faire, nous lui créons une websocket vers le serveur sur le port 443 et nous renvoyons un 200 OK au client pour signifier à ce dernier qu'il peut envoyer ses données. \linebreak
Ensuite, lors de la phase d'échange, lorsque nous recevons une donnée du serveur, nous la transmettons immédiatement au client. Et lorsque l'on reçoit un paquet de type inconnu de la part d'un client, on renvoie directement et sans traitement le paquet au serveur. En effet, étant donnée que les données sont cryptées, nous ne pouvons récupérer aucune information sur l'échange.